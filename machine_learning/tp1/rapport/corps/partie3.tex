\section{Questions}

\begin{enumerate}
    \item \textbf{Définissez les termes:}
    \begin{itemize}
        \item \textbf{Approches supervisées} \\
        L'approche supervisées est utilisé quand on à un certains nombres de données connus et correcte. C'est données sont utilisé pour entrainer le modèle pour que celui-ci puisse nous 
        donner une prévision proche de la réalité. Nous avons donc appliqué une approche supervisée dans ce TP.

        \item \textbf{Approches non supervisées} \\
        Il s'agit de l'inverse de l'approche supervisée, ici on lui demande un résultat à l'aide de données alèatoire.

        \item \textbf{Régression} \\
        Utilisé pour déterminer une prédiction optimal à l'aide d'une relation linéaire entre unr variable dépendante et plusieurs variables indépendantes.

        \item \textbf{Classification} \\
        Utilisé pour déterminer à qu'elle classe appartient une observation à partir de précédentes connus.

    \end{itemize}
    \item \textbf{Représenter en un schéma général, les processus d'apprentissage et de prédiction ?} \\
    \item \textbf{Comment fonctionne l'apprentissage ? Par quels moyens ? A quoi sert la fonction de coût ? Comment est-résolu le problème ? Connaissez vous d'autres moyens de le résoudre ?} \\
    La méthode d'apprantissage par régression linéaire fonctionne grâce à l'ajustement du paramètre de notre modèle $\theta$. Pour ce faire il faut minimiser la fonction de coût \textit{(erreur entre la prédiction et la réalité)} à l'aide
    d'une descente de gradient ce qui permet d'obtenir un $\theta$ pour réaliser une prédiction optimal. \\
    On peut résoudre ce problème avec une descente de gradient, mais également avec les équations normal.

    \item \textbf{Pourquoi faut-il parfois normaliser les descripteurs (features) ?} \\
    Il est intéressant de normaliser les données pour les ramener à une échelle comune, 
    ce qui permet d'interpréter plus facilement les résultat et rendre les calculs plus stable et rapide. Grâce à la normalisation on peut être sûr que chaque caractéristiques contribue équitablement à la prédiction du modèle, indépendamment de son échelle initiale. \\
    
\end{enumerate}