\section{Constitution d’un ensemble de données d’apprentissage}

\textbf{Question 1: Sous-échantillonage}

\begin{figure}[!h]
    \begin{minted}[frame=lines, framesep=2mm, baselinestretch=1.2, fontsize=\footnotesize, linenos, breaklines=true]{python}
def preprocessing():
    # [...]
    feat, nbPix, nbFeat = selectFeatureVectors(img73, 500)
    print(f"Number of training data: {nbPix}")
    
    return feat, img73, img87
    """return
    Number of training data: 1714 
    """
    \end{minted}   
    \captionof{listing}{\label{lst:main}Main Function}
\end{figure}

Le résultat obtenu est le nombre de données d'apprentissage, il est donc de 1714. En sous-échantillonnant les 
données, on va réduire la taille de l'ensemble d'apprentissage, ce qui va nous simplifier la tâche pour la suite. 


