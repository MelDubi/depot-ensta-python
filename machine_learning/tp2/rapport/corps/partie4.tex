\clearpage
\section{Questions}


\begin{enumerate}
    \item \textbf{Identifiez les différences d'approche entre le TP1 et le TP2 ? et les différences de résolution du problème ?}\\
    L'objectif entre c'est deux TP est différent. \\
    Dans le premier TP nous avons un historique de données à partir du quel on construit notre modèle, la finalité et de pouvoir déterminer par exemple le prix d'une maison à partir de ses caractéristiques. Caractéristiques qui 
    font parties de notre modèle. \\
    Dans ce second TP nous avons également un modèle fondé sur un historique de données, mais cette fois-ci nous avons des classes. L'objectif et de prédire à quelle classe appartient notre échantillon.\\
    
    Ce sont des méthodes différentes qui s'appliquent dans des cas dont l'objectif n'est pas le même.

    \vspace{0.5cm}
    
    \item \textbf{A quoi sert le principe de régularisation ?} \\
    La régularisation permet de prévenir les problèmes de sur-apprentissage \textit{(overfit)}. Un problème expliqué à la question suivante et comment y remédier à l'aide de la régularisation.
    \vspace{0.5cm}

    \item \textbf{Décrivez le problème du sur-apprentissage et comment on y répond dans ce TP. Quelle est l'influence de la valeur de $\lambda$ ?} \\
    Le problème de sur-apprentissage est lié à la question précédente. Il survient dû à un nombre important de caractéristiques et à leurs valeurs. Pour les atténuées on utilise la régulation, ce qui permet d'obtenir un 
    modèle moins complexe et adapté pour obtenir des prédictions fiables. \\
    Il est important de choisir correctement $\lambda$, si celui-ci est trop important alors les caractéristiques seront fortement atténuées. On ne fera plus face à un problème de sur-apprentissage, mais de sous-apprentissage.
    \vspace{0.5cm}

    \item \textbf{Dans quelles circonstances et pour quelles raisons utilise-t-on l'approche multiclasse ? Quelles sont les différentes possibilités pour cette approche ?} \\
    Quand il faut simplement répondre à une question par \textit{oui} ou \textit{non}, nous pouvons simplement utilisé une approche monoclasse. En revanche quand la réponse peut avoir plus de deux possibilités comme \textit{0, 1, 2, 3, 4}
    ou encore \textit{rouge, orange, vert} alors une approche multiclasse est nécessaire. \\
    Pour mettre en place cette approche nous pouvons utiliser la méthode \textit{One-Vs-All}, \textit{One-Vs-One}...

\end{enumerate}
