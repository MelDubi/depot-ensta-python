\clearpage
\section{Questions}

\begin{enumerate}
    \item \textbf{Décrivez l’approche basée sur les réseaux de neurones. Quelle modélisation est retenue? Quels sont les paramètres à apprendre?}\\
    
    Les réseaux de neurones fonctionnent avec plusieurs couches, la couche d'entrée, les couches cachées et la couche de sortie.
    Le but est toujours le même, favorisé l'apprentissage automatique de la machine. Les couches sont connectées entre elles. 
    A chaque couche, des calculs sont effectués et les informations sont transmises aux couches suivantes. Elles sont composées de neurones
    qui correspondent en entrée aux caractéristiques des données que l'on inserent au réseau. Dans les couches cachées, les neurones
    apprennent des informations de plus en plus compliquées. La couche de sortie renvoit le résultat du réseau de neurone. \\

    Les paramètres en jeu dans cette approche sont les poids de connexions entre les couches et les biais. L'objectif dans un réseau
    de neurone est d'ajuster les poids et les biais pour minimiser le coût. Les poids correspondent à la connexion entre les neurones, 
    dans le sens où l'information transmise entre deux neurones sera plus ou moins importante ou non pour le prochain neurone. Les biais
    quant à eux, influencent l'activation d'un neurone et permettent d'apprendre plus facilement des modèles complexes. 


    \vspace{0.5cm}
    
    \item \textbf{Quelle est la fonction de coût? Quelles modifications sont effectuées par rapport à celle utilisée en TP2?} \\
    
    La fonction de coût pour le réseau de neurone nous sert à évaluer la qualité de la prédiction. La différence ici, est qu'à 
    l'aide de la rétropropagation, les paramètres du réseau vont être ajustés à chaque échantillon en fonction du poids calculé
    en sortie. \\

    Les différences entre les deux fonctions de coût des deux TP2 sont principalement dans la compléxité de la formule dû aux couches
    multiples du réseau de neurone. En effet, pour ce TP3, le réseau de neurone s'appuie sur deux $\theta_1$ contre un seul dans le TP2.
    A la finale, l'objectif reste toujours le même, minimiser l'erreur de prédiction par rapport aux valeurs réelles attendues. 

\end{enumerate}